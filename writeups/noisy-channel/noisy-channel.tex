\documentclass[11pt]{article}
\usepackage[margin=1.5in]{geometry}
\setlength\parindent{0pt}
\setlength\parskip{0.2cm}
\usepackage{amsmath}


\begin{document}

\begin{center}
	
	{\Large \textbf{Chart-based noisy channel PLU model}}

	{\normalsize Emily Kellison-Linn}
		
	{\footnotesize 8 Aug 2017}
	
\end{center}

\section{Model}

The following describes a noisy channel model for computing a distribution over bottom-level PLU sequences given a top-level PLU sequence and a set of model parameters.

\subsection{Parameters}

Let:

\begin{itemize}

\item $a_1, ... a_k \equiv$ the alphabet of $k$ unique PLUs

\item $D_1, ... D_k \equiv$ the deletion probabilities for each PLU. $D_i$ corresponds to the probability of top-level PLU $a_i$ being deleted and thus not present in the bottom level at any given index at which $a_i$ appears in $s$. $D_i\in [0,1]~\forall i$. 

\item $I_1, ... I_k \equiv$ the insertion probabilities for each PLU. $I_i$ corresponds to the probability of top-level PLU $a_i$ being inserted into the bottom level at any given index. $I_i\in [0,1]~\forall i$.

\item $S_{11}, ... S_{kk} \equiv$ the substitution probabilities for each pair of PLUs. $S_{i,j}$ corresponds to the probability of top-level PLU $a_i$ being replaced by PLU $a_j$ in the bottom level at any given index at which $a_i$ appears.$S_{i,j}\in [0,1]~\forall i,j$.

\item $P_{11}, ... P_{nk} \equiv$ the prior probabilities of each PLU at each index, based on the properties of the acoustic data.

\end{itemize}

\subsection{Data}

Let $s_1, ... s_n \equiv$ the top-level PLU sequence of length $n$, \\ where $s_i \in \{a_1, ... a_x\}$ for all $1\leq i\leq n$.

\subsection{Computation}

Let $M_{111}, ... M_{n,n,2k+1} \equiv$ the PLU sequence probability chart. \textbf{Note: I know we should actually permit $M$ to be of size $M_{n,m,2k+1}$, where $m$ is some number slightly larger than $n$. But under my current understanding of the model I don't understand how that works yet.}


Each 1-dimensional vector $M_{ij*}$ of $M$ corresponds to a prefix $s_1, ... s_i-1$ of $s$ and a prefix of length $j-1$ of the bottom-level PLUs. Each individual cell $M_{ijq}$ corresponds to the probability of the most probable bottom-level PLU sequence of length $j-1$ generated from $s_1, ... s_i-1$ whose last edit operation is $q$.

For each $M_{ij*}$, let the first element correspond to a delete operation (deleting $s_i-1$); let the next $k$ elements correspond to insert operations (inserting $a_1, ... a_k$ following $s_{i-2}$); and let the next $k$ elements correspond to substitution operations (substituting $a_1, ... a_k$ for $s_i-1$). This gives us $|M_{ij*}|=2k+1$.

The chart can then be filled iteratively as follows:

%$M_{abc} = 0 ~\forall a,b,c$ \hfill (initialize to 0)

$M_{11q} = 1 ~\forall q$ \hfill (probabilities for transforming one 0-length string into another)


$M_{1,j,q} = \begin{cases}
		{max}(M_{1,j-1,*}) \times I_q' \times P_{j-1,q'} ~;~ 2\leq q\leq k+1 \text{ where } q'=q-(k+1) \\
		0 \text{ otherwise} \\
		\end{cases}$ \\\-\hfill (first row; series of insertions)

$M_{i,1,q} = \begin{cases}
		{max}(M_{i-1,1,*}) \times D_{s_{i-1}} ~;~ q=1 \\
		0 \text{ otherwise} \\
		\end{cases}$ \\\-\hfill (first column; series of deletions)
		
For the general case ($i>1$ and $j>1$):
		
$M_{i,j,q} = \begin{cases}
		{max}(M_{i-1,j,*}) \times D_{s_{i-1}} ~;~ q=1 \hfill\text{(Delete operation)}\\
		{max}(M_{i,j-1,*}) \times I_q' \times P_{j-1,q''} ~;~ 2\leq q\leq k+1 \text{ where } q''=q-1 \\\-\hfill\text{(Insert operations)}\\
		{max}(M_{i-1,j-1,*}) \times S_{s_{i-1},q'} \times P_{j-1,q'} ~;~ k+2\leq q\leq 2k+1 \text{ where } q'=q-(k+1) \\\-\hfill\text{(Substitute operations)}\\
		\end{cases}$ \\\-\hfill


\end{document}
